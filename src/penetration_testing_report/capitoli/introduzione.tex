\phantomsection
%\addcontentsline{toc}{chapter}{Introduzione}
\chapter{Introduzione}
\markboth{Introduzione}{}
% [titolo ridotto se non ci dovesse stare] {titolo completo}
Il presente documento ha l'obiettivo di illustrare le metodologie utilizzate nell'ambito dell'attività progettuale svolta nel contesto del corso di \emph{Penetration Testing and Ethical Hacking}, tenuto dal prof. Arcangelo Castiglione presso l'Università degli studi di Salerno durante l'anno accademico 2022/2023. L'attività progettuale in questione consiste nello svolgimento del processo di \emph{Penetration Testing} su un asset vulnerabile \emph{by-design}; nello specifico, è stata scelta la macchina virtuale \emph{Momentum: 1} messa a disposizione sulla piattaforma \emph{VulnHub} dall'utente \emph{AL1ENUM}. 

\section{Processo di \emph{Penetration Testing}}
