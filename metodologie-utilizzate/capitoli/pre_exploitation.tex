\phantomsection

\chapter{Pre-Exploitation}
\markboth{Pre-Exploitation}{}

\begin{citazione}
\end{citazione}
\newpage

\section{Target Scoping} { \setstretch{1.3}}
Dal momento che il processo di \emph{Penetration Testing} ha uno scopo puramente didattico non è prevista una fase di accordo tra le parti coinvolte in quanto l'asset da analizzare è una macchina virtuale vulnerabile \emph{by design}. Non vi è, infatti, un cliente dal quale raccogliere requisiti e con il quale definire obiettivi di business e modelli dei costi. Il processo verrà svolto senza particolari vincoli formali relativi all'asset.

\section{Information Gathering} { \setstretch{1.3}}
La caratterizzazione dell'asset da analizzare può generalmente avvenire mediante molteplici \emph{tool} e coinvolgere diversi aspetti dell'asset stesso. Dal momento che si sta trattando una macchina virtuale vulnerabile \emph{by-design} contestualizzata in un'attività progettuale avente uno scopo didattico non risulta utile ricorrere a particolari tecniche \emph{OSINT (Open Source INTelligence)}, né a tecniche volte all'ottenimento di informazioni di routing e record DNS. Sono state, tuttavia, consultate le informazioni di base dell'asset disponibili sulla piattaforma \emph{VulnHub} che mette a disposizione la macchina virtuale. Le informazioni fornite sono le seguenti:
\begin{itemize}
    \item \textbf{Nome della macchina}: \emph{Momentum: 1};
    \item \textbf{Sistema Operativo}: \emph{Linux};
    \item \textbf{DHCP Server}: abilitato;
    \item \textbf{Indirizzo IP}: assegnato in automatico.
\end{itemize}
Non risultano, dunque, note le informazioni relative all'indirizzo IP della macchina né le credenziali di accesso alla stessa. 
\section{Target Discovery} { \setstretch{1.3}}

\section{Target Enumeration} { \setstretch{1.3}}

\section{}